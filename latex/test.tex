\documentclass[aspectratio=169]{beamer} %[handout]
\mode<presentation>
\setbeamercovered{transparent}
\usetheme[hideothersubsections]{Hannover}
\usecolortheme{fly}
\usenavigationsymbolstemplate{}
\usepackage{lmodern}
\renewcommand{\familydefault}{\sfdefault}

%%language
\usepackage[utf8]{inputenc}
\usepackage[T1]{fontenc}
\usepackage{ngerman}
\usepackage[ngerman,iso]{isodate}
%\usepackage{eurosym}
\hyphenation{}

\setbeamertemplate{section in toc}{\hspace*{1em}\inserttocsection}

\usepackage{graphicx}
\usepackage{multicol}
\usepackage{wallpaper}

%\usebackgroundtemplate{\only<2->{
%    \rotatebox{90}{\makebox[5.7cm][l]{}}
%    \makebox[\textwidth][r]{\includegraphics[height=2.6cm]{pic/FFW_Wassel.jpg}}
%}}
\begin{document}
\title{Feinstaub in Lehrte}
\subtitle{Projektausschnitt von \href{http://www.luftdaten.info}{luftdaten.info}}
\author{Dr. Hans-Jürgen Dankert, Sebastian Frenger}
\institute{luftdaten.info} 
\date{2018-01-23} 

\frame{\titlepage} 

\begin{frame}{Inhalt}
  \begin{multicols}{2}
    \tableofcontents
  \end{multicols}
\end{frame}

\section{Das Projekt}  
\begin{frame}{Stuttgart}
  \begin{itemize}
  \item open Data
  \item Stuttgart im Tal
  \item Feinstaub messen: teuer?
  \end{itemize}
\end{frame}

\begin{frame}{Interessant für Lehrte?}
  \begin{itemize}
  \item Kleingärten
  \item grüne Lunge
  \item soll Baugebiet weichen
  \item verträglich?
  \end{itemize}
\end{frame}

\begin{frame}{Teilnehmer}
  \begin{itemize}
  \item Aktionsbündnis Lehrter Laubenpierer
  \item DIE LINKE. Lehrte/Sehnde
  \item Piratenpartei Lehrte
  \item Siedlergemeinschaft Hohnhorst Lehrte e.V.
  \item weitere Anlieger
  \end{itemize}
\end{frame}

\subsection{September}
\begin{frame}{braunschweig}
  \begin{center}
    \includegraphics[width=\textwidth]{../gnuplot/braunschweig-2017-09.png}
  \end{center}
\end{frame}
\begin{frame}{hannover}
  \begin{center}
    \includegraphics[width=\textwidth]{../gnuplot/hannover-2017-09.png}
  \end{center}
\end{frame}
\begin{frame}{lehrte}
  \begin{center}
    \includegraphics[width=\textwidth]{../gnuplot/lehrte-2017-09.png}
  \end{center}
\end{frame}

\subsection{Oktober}
\begin{frame}{braunschweig}
  \begin{center}
    \includegraphics[width=\textwidth]{../gnuplot/braunschweig-2017-10.png}
  \end{center}
\end{frame}
\begin{frame}{hannover}
  \begin{center}
    \includegraphics[width=\textwidth]{../gnuplot/hannover-2017-10.png}
  \end{center}
\end{frame}
\begin{frame}{lehrte}
  \begin{center}
    \includegraphics[width=\textwidth]{../gnuplot/lehrte-2017-10.png}
  \end{center}
\end{frame}

\subsection{November}
\begin{frame}{braunschweig}
  \begin{center}
    \includegraphics[width=\textwidth]{../gnuplot/braunschweig-2017-11.png}
  \end{center}
\end{frame}
\begin{frame}{hannover}
  \begin{center}
    \includegraphics[width=\textwidth]{../gnuplot/hannover-2017-11.png}
  \end{center}
\end{frame}
\begin{frame}{lehrte}
  \begin{center}
    \includegraphics[width=\textwidth]{../gnuplot/lehrte-2017-11.png}
  \end{center}
\end{frame}

\subsection{Dezember}
\begin{frame}{braunschweig}
  \begin{center}
    \includegraphics[width=\textwidth]{../gnuplot/braunschweig-2017-12.png}
  \end{center}
\end{frame}
\begin{frame}{hannover}
  \begin{center}
    \includegraphics[width=\textwidth]{../gnuplot/hannover-2017-12.png}
  \end{center}
\end{frame}
\begin{frame}{lehrte}
  \begin{center}
    \includegraphics[width=\textwidth]{../gnuplot/lehrte-2017-12.png}
  \end{center}
\end{frame}

\subsection{Januar}
\begin{frame}{braunschweig}
  \begin{center}
    \includegraphics[width=\textwidth]{../gnuplot/braunschweig-2018-01.png}
  \end{center}
\end{frame}
\begin{frame}{hannover}
  \begin{center}
    \includegraphics[width=\textwidth]{../gnuplot/hannover-2018-01.png}
  \end{center}
\end{frame}
\begin{frame}{lehrte}
  \begin{center}
    \includegraphics[width=\textwidth]{../gnuplot/lehrte-2018-01.png}
  \end{center}
\end{frame}

\end{document}
